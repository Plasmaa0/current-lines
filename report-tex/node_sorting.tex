\subsection{Нумерация узлов КЭ}
Для корректной работы метода векторных произведений для определения положения точки относительно КЭ требуется, чтобы узлы КЭ были перечислены в порядке по часовой стрелке. При разработке алгоритма было принято определение ребра, как двух подряд идущих вершин. Если вершины будут пронумерованы не по порядку, то при обработке ребра алгоритм на самом деле будет обрабатывать отрезок, лежащий внутри КЭ, а не ребро.

Формат \verb|.msh| не гарантирует, что ребра пронумерованы в соответствие с определением ребра выше. 

Необходимо реализовать специальный алгоритм сортировки вершин КЭ по часовой стрелке. Ключевая вещь, необходимая алгоритму сортировки -- это возможность сравнить два элемента массива. Следовательно нужно создать функцию, которая будет сравнивать две точки и расставлять их по порядку по часовой стрелке.

\subsubsection{Сравнение двух точек}
\lstinputlisting[language=C++, firstline=21, lastline=52]{current-lines/src/Element/Element.cpp}
Точки сортируются относительно угла $\frac{\pi}{2}$ по часовой стрелке. Началом координат будеv считать предварительно заданную точку $C$ -- центр. 

\paragraph{Простые случаи}
Сперва сравним координаты $x$ точек, если они находятся в разных полуплоскостях относительно центра, то можно сразу сделать вывод о том, какая из них больше.

Если точки находятся на вертикальной прямой $x=C_x$, то сравним их по удаленности от центра $|y-C_y|$.

\paragraph{Общий случай}
Посчитаем векторное произведение $\vec{CA}\times\vec{CB}$ и в зависимости от его знака определим относительное положение точек $A$ и $B$ (аналогично тому, как это было сделано при определении принадлежности точки многограннику).

\paragraph{Точки с одинаковым значением угла}
Для точек с одинаковым значением угла векторное произведение $\vec{CA}\times\vec{CB}$ будет равным $\vec{0}$, тогда сравним их по удалённости от центра $|CA|, |CB|$.

\paragraph{О производительности}
Для сравнения точек можно было применить также функцию $\operatorname{atan2}$, определяемую следующим образом
\begin{equation*}
	\operatorname{atan2}(y,x)=\begin{cases}
			\arctan(\frac{y}{x}), &x>0,\\
			\arctan(\frac{y}{x})+\pi, &x<0 \wedge y\geq 0,\\
			\arctan(\frac{y}{x})-\pi, &x<0 \wedge y<0,\\
			+\frac{\pi}{2}, &x=0 \wedge y>0,\\
			-\frac{\pi}{2}, &x=0 \wedge y<0,\\
			\varnothing, &x=0\wedge y=0.
		\end{cases}
\end{equation*}
Одним лишь сравнением значений $\operatorname{atan2}$ для точек $A$ и $B$ можно их упорядочить. Но данная функция затрачивает намного больше вычислительных возможностей компьютера, поскольку использует деление чисел с плавающей точкой и вычисление тригонометрических функций. Предложенное ранее сравнение двух точек использует только операции сложения(вычитания) и умножения, что делает её намного производительнее.

\subsubsection{Сортировка}
Испольуется алгоритм сортировки \verb|std::sort| из стандартной библиотеки \verb|C++|, центр сортировки вычисляется предварительно как среднее арифметическое координат всех вершин КЭ.
\lstinputlisting[language=C++, firstline=65, lastline=75]{current-lines/src/Element/Element.cpp}
