% следующие 3 строчки нужны для точечек в содержании
\usepackage{tocloft}
\renewcommand{\cftpartleader}{\cftdotfill{\cftdotsep}}
\renewcommand{\cftchapleader}{\cftdotfill{\cftdotsep}}
\usepackage{subcaption}
\usepackage{fancyvrb}
% шапка в титульнике чтоб лучше стала
\renewcommand{\titlepageheader}[2]
{
	\begin{wrapfigure}[7]{l}{0.14\linewidth}
		\vspace{3.4mm}
		\hspace{-8mm}
		\includegraphics[width=0.89\linewidth]{bmstu-logo}
	\end{wrapfigure}
	
	{
		\singlespacing \small
		Министерство науки и высшего образования Российской Федерации \\
		Федеральное государственное бюджетное образовательное учреждение \\
		высшего образования \\
		<<Московский государственный технический университет \\
		имени Н.~Э.~Баумана \\
		(национальный исследовательский университет)>> \\
		(МГТУ им. Н.~Э.~Баумана) \\
	}
	
	\vspace{-4.2mm}
	\vhrulefill{0.9mm} \\
	\vspace{-7mm}
	\vhrulefill{0.2mm} \\
	\vspace{2.8mm}
	
	{
		\small
		ФАКУЛЬТЕТ \longunderline{<<#1>>} \\
		\vspace{3.3mm}
		КАФЕДРА \longunderline{#2} \\ % вот здесь убрал <<>>, чтобы
		% можно было подписать (ФН-11) вне кавычек 
	}
}

% Создание титульной страницы РПЗ к КР
\renewcommand{\makecourseworktitle}[5]
{
	\documentmeta{РПЗ к КР}{#4}{}{#3}
	
	\begin{titlepage}
		\centering
		
		\titlepageheader{#1}{#2}
		\vspace{15.8mm}
		
		\titlepagenotetitle{К КУРСОВОЙ РАБОТЕ}{#3}
		
		% следующие строчки добавлены мной, чтобы отображать дисциплину
		\vspace{10mm}
		\raggedright \large
		Дисциплина: \underline{Дифференциальные уравнения}
		
		\centering
		% здесь мои строчки заканчиваются
		
		\vfill
		
		\titlepageauthors{#4}{Руководитель курсовой работы,\\канд. физ.-мат. наук, доцент}{#5} % {#6}{#7}
		
		% \vspace{14mm}
		
		\vspace{14mm}
		\raggedright \small
		Оценка: \rule{4cm}{0.15mm}
		
		%\vspace{14mm}
		
		\centering
		
		\textit{Москва, {\the\year} г.}
	\end{titlepage}
	
	\setcounter{page}{2}
}

% Установка исполнителей работы
\renewcommand{\titlepageauthors}[5]
{
	{
		\renewcommand{\titlepagestudentscontent}{}
		\maketitlepagestudent{#1}
		
		\renewcommand{\titlepageotherscontent}{}
		\maketitlepageothers{#2}{#3}
		%\maketitlepageothers{Консультант}{#4}
		%\maketitlepageothers{Нормоконтролер}{#5}
		
		\small
		\begin{tabularx}{\textwidth}{@{}>{\hsize=.5\hsize}X>{\hsize=.25\hsize}X>{\hsize=.25\hsize}X@{}}
			\titlepagestudentscontent
			
			\titlepageotherscontent
		\end{tabularx}
	}
}

% для второй странички
\newcommand{\lineup}[5]{
	\newcommand*\lineupothercontent{}
	
	\newcommand{\makelineupother}[5]
	{
		\foreach \c in {#3} {
			\gappto\lineupothercontent{#2 &}
			\gappto\lineupothercontent{\fixunderline{}{40mm}{(Подпись, дата)} \vspace{1.3mm} &}
			\gappto\lineupothercontent{\fixunderline}
			\xappto\lineupothercontent{{\c}}
			\gappto\lineupothercontent{{40mm}{(И.~О.~Фамилия)} \\}
		}
		
		\foreach \s/\g in {#1} {
			\gappto\lineupothercontent{Исполнитель,\\студент группы \fixunderline}
			\xappto\lineupothercontent{{\g}}
			\gappto\lineupothercontent{{25mm}{(Группа)} &}
			\gappto\lineupothercontent{\fixunderline{}{40mm}{(Подпись, дата)} \vspace{1.3mm} &}
			\gappto\lineupothercontent{\fixunderline}
			\xappto\lineupothercontent{{\s}}
			\gappto\lineupothercontent{{40mm}{(И.~О.~Фамилия)} \\}
		}
		
		\foreach \c in {#5} {
			\gappto\lineupothercontent{#4 &}
			\gappto\lineupothercontent{\fixunderline{}{40mm}{(Подпись, дата)} \vspace{1.3mm} &}
			\gappto\lineupothercontent{\fixunderline}
			\xappto\lineupothercontent{{\c}}
			\gappto\lineupothercontent{{40mm}{(И.~О.~Фамилия)} \\}
		}
		
	}
	
	\makelineupother{#1}{#2}{#3}{#4}{#5}
	
	\centerline{\MakeUppercase{\textbf{Список исполнителей}}}
	
	\vspace{14mm}
	
	{\small\raggedright \begin{tabularx}{\textwidth}{@{}>{\hsize=.5\hsize}X>{\hsize=.25\hsize}X>{\hsize=.25\hsize}X@{}}
			\lineupothercontent
	\end{tabularx}}
	
	\pagebreak
}

\newif\ifgdeSep
\newcommand*{\gde}[1]{%
	\begin{itemize}[noitemsep]
		\gdeSepfalse
		\item[где]
		\gdeScan#1\relax \text{.}
	\end{itemize}
}
\newcommand{\gdeScan}[2]{%
	\ifx\relax#1\empty
	\else
	\ifgdeSep
	;\item[]\relax
	\else
	\gdeSeptrue
	\fi
	#1 -- #2\relax
	\expandafter\gdeScan
	\fi
}


\AddEnumerateCounter{\asbuk}{\russian@alph}{щ}

%\let\oldItemize\itemize
\newenvironment{spisok}
{\begin{itemize}[label=---,itemindent=\parindent,leftmargin=0pt]}
	{\end{itemize}}

\newenvironment{ruspisok}
{\begin{enumerate}[label=\arabic*),itemindent=\parindent*2,leftmargin=0pt,ref=\alph*]}
	{\end{enumerate}}

\renewcommand{\maketableofcontents}
{
	\setlength{\cftbeforetoctitleskip}{-2em}
	\setlength{\cftaftertoctitleskip}{0em}
	\renewcommand\contentsname{
		\normalsize \centerline{\MakeUppercase{Содержание}}\hfill c.
	}
	%\singlespacing
%	\begin{singlespace}
		\tableofcontents
%	\end{singlespace}
}


\newcommand{\makepraktikareporttitle}[7]
{
	\documentmeta{Отчет}{}{}{}
	
	\begin{titlepage}
		\centering
		
		\titlepageheader{#1}{#2}
		\vspace{15.8mm}
		
		\large
		\textbf{ОТЧЕТ ПО ПРАКТИКЕ}
		
		\foreach \s/\g in {#3} {
			
			\vspace{10mm}
			
			\raggedright \small
			
			Cтудент \longunderline{#7}
			
			\vspace{5mm}
			
			Группа \longunderline{\g}
			
			\vspace{5mm}
			
			Тип практики \longunderline{#5}
			
			\vspace{2mm}
			
			Название предприятия \longunderline{#6}
		}
		\vfill
		
		\titlepageauthors{#3}{Руководитель практики}{#4}{}{}
		\raggedright \small
		Оценка: \rule{4cm}{0.15mm}
		
		%\vspace{14mm}
		
		%		\raggedright \small
		%		Оценка: \rule{4cm}{0.15mm}
		
		\vspace{14mm}
		
		\centering
		\textit{Москва, {\the\year} г.}
	\end{titlepage}
	
	\setcounter{page}{2}
}

%\DeclareMathOperator{\ch}{ch}
%\DeclareMathOperator{\sh}{sh}
%\DeclareMathOperator{\Arctgh}{arctgh}
%
%\DeclareMathOperator{\Arctg}{arctg}
\usepackage{color}

\definecolor{codegreen}{rgb}{0,0.6,0}
\definecolor{codegray}{rgb}{0.5,0.5,0.5}
\definecolor{codepurple}{rgb}{0.58,0,0.82}
\definecolor{backcolour}{rgb}{1,1,1}

\lstdefinestyle{mystyle}{
	backgroundcolor=\color{backcolour},
	commentstyle=\color{codegreen},
	keywordstyle=\color{magenta},
	numberstyle=\tiny\color{codegray},
	stringstyle=\color{codepurple},
	basicstyle=\ttfamily\footnotesize,
	breakatwhitespace=false,
	breaklines=true,
	captionpos=b,
	keepspaces=true,
	numbers=left,
	numbersep=5pt,
	showspaces=false,
	showstringspaces=false,
	showtabs=false,
	tabsize=4,
	texcl=true,
	emph={trunc, log, histogram, None, plot, xlabel, ylabel, hist, True, mean, std, sqrt, linspace, norm, pdf
		, flatten, gamma, as, comb, uniform, sort, var, floor},
	emphstyle=\color{magenta}
}

\lstset{style=mystyle}